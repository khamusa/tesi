%
% Tesi D.S.I. - modello preso da
% Stanford University PhD thesis style -- modifications to the report style
%
%%%%%%%%%%%%%%%%%%%%%%%%%%%%%%%%%%%%%%%%%%%%%%%%%%%%%%%%%%%%%%%%%%%%%%%%%%%
%                                                                         %
%			TESI DOTTORATO                                                   %
%			______________                                                   %
%                                                                         %
%			AUTORE: Elena Pagani                                             %
%                                                                         %
%			Ultima revisione: 7.X.1998                                       %
%                                                                         %
%%%%%%%%%%%%%%%%%%%%%%%%%%%%%%%%%%%%%%%%%%%%%%%%%%%%%%%%%%%%%%%%%%%%%%%%%%%
%
%
\documentclass[12pt]{report}
%
% \includeonly{}
%
%			PREAMBOLO
%
\usepackage[a4paper]{geometry}
\usepackage{amssymb,amsmath,amsthm}
\usepackage{graphicx}
\usepackage{url}
\usepackage{hyperref}
\usepackage{epsfig}
\usepackage[italian]{babel}
\usepackage{tesi}

% per le accentate
\usepackage[utf8]{inputenc}

%
\newtheorem{myteor}{Teorema}[section]
%
\newenvironment{teor}{\begin{myteor}\sl}{\end{myteor}}
%
%
%			TITOLO
%
\begin{document}
\title{Progettazione di un'applicazione e conseguenze di un approccio ``continuos refactoring''}
\author{Samuel Gomes Brandão}
\dept{Corso di Laurea in Informatica} 
\anno{2013-2014}
\matricola{803939}
\relatore{Prof. Carlo Bellettini}
%\correlatore{Dr. Richard STEVENS}
%
%        \submitdate{month year in which submitted to GPO}
%		- date LaTeX'd if omitted
%	\copyrightyear{year degree conferred (next year if submitted in Dec.)}
%		- year LaTeX'd (or next year, in December) if omitted
%	\copyrighttrue or \copyrightfakblse
%		- produce or don't produce a copyright page (false by default)
%	\figurespagetrue or \figurespagefalse
%		- produce or don't produce a List of Figures page
%		  (false by default)
%	\tablespagetrue or \tablespagefalse
%		- produce or don't produce a List of Tables page
%		  (false by default)
% 
%			DEDICA
%
\beforepreface
\prefacesection{}
        {\hfill \Large {\sl Ai miei genitori Mônica e Zilmar}}
% 
%			PREFAZIONE
%
\prefacesection{Prefazione}
Ciao! Ci vuole scrivere una prefazione!
%
%
%			ORGANIZZAZIONE
\section*{Organizzazione della tesi}
\label{organizzazione}
La tesi \`e organizzata come segue:
\begin{itemize}
\item nel Capitolo 1 ....
\end{itemize}
%
%			RINGRAZIAMENTI
%
\prefacesection{Ringraziamenti}
Vorrei ringraziare a Paolo Venturi e Diego Costantino, 
al prof. Dr. Carlo Bellettini. 
\afterpreface


% 
% 
%			CAPITOLO 1: iNTRODUZIONE
\chapter{Introduzione}
\label{cap1}
Spazi Unimi è il nome dato a un progetto per l’ottenimento di dati sugli spazi dell’Università degli Studi di Milano, sviluppato durante il Tirocinio Interno per la laurea triennale in Informatica all’UNIMI, da Samuel Gomes Brandão, Diego Costantino e Palo Venturi. L’idea nasce a partire dalle proposte del progetto Campus Sostenibile, promosso dall’UNIMI e dal Politecnico di Milano, e si sviluppa posteriormente in autonomia, sotto l’orientamento del Prof. Dr. Carlo Bellettini.

Il progetto parte con lo sviluppo di un'applicazione software con lo scopo principale di estrarre, validare e correggere dati ottenuti da diverse sorgenti, procedendo in seguito alla loro integrazione. I dati vengono mantenuti su un database da venir utilizzato per futuri progetti e sviluppi, attraverso l’uso di una specifica Application Programming Interface (API) con architettura REST (Representational State Transfer). 

Le informazioni integrate riguardano la topologia e la destinazione d'uso degli edifici universitari e i loro spazi, con particolare importanza alla elaborazione e presentazione delle piante interne e localizzazione di stanze precise. In questo modo, siamo in grado di fornire accuratamente informazioni sulla localizzazione di palazzi, aule, o stanze a secondo della loro tipologia d'uso (bagni, biblioteca, sale studio, ecc).

Su questa relazione descrivo il processo di sviluppo della suddetta applicazione da zero, con particolare rilievo alle sfide per la buona progettazione e all’utilizzo di un concetto che ho chiamato “continuos refactoring” \footnote{Si veda il capitolo \ref{cap3}}



% 
% 
%			CAPITOLO 2
\chapter{Attività Preliminari}
\label{cap2}
Lorem Ipsum dolor sit



% 
% 
%			CAPITOLO 3
\chapter{Svolgimento delle Attività}
\label{cap3}
Lorem Ipsum dolor sit


% 
% 
%			CAPITOLO 3.1 ? Come fare sottocapitoli, fuck?
\section{Refactoring Guidato - SOLID}
Lorem Ipsum dolor sit



%
%

%
%			BIBLIOGRAFIA
%
\begin{thebibliography}{00}
%
\bibitem{gotti91}
M. Gotti, I linguaggi specialistici, Firenze, La Nuova Italia, 1991.
%
\bibitem{wellek62}
R. Wellek, A. Warren, Theory of Literature , 3rd edition, New York, Harcourt, 1962.
%
\bibitem{canziani78}
A. Canziani et al., Come comunica il teatro: dal testo alla scena. Milano, Il Formichiere, 1978.
%
\bibitem{MoD67}
Ministry of Defence, Great Britain, Author and Subject Catalogues of the Naval Library, London, Ministry of Defence, HMSO, 1967.
%
\bibitem{heine23}
H. Heine, Pensieri e ghiribizzi. A cura di A. Meozzi. Lanciano, Carabba, 1923.
%
\bibitem{basso62}
L. Basso, ``Capitalismo monopolistico e strategia operaia'', Problemi del socialismo, vol. 8, n. 5, pp. 585-612, 1962.
%
\bibitem{avirovic93}
L. Avirovic, J. Dodds (a cura di), Atti del Convegno internazionale "Umberto Eco, Claudio Magris. Autori e traduttori a confronto" ( Trieste, 27-28 novembre 1989), Udine, Campanotto, 1993.
%
\bibitem{gans67}
E.L. Gans, "The Discovery of Illusion: Flaubert's Early Works, 1835-1837", unpublished Ph.D. Dissertation, Johns Hopkins University, 1967.
%
\bibitem{harrison92}
R. Harrison, Bibliography of planned languages (excluding Esperanto).  \url{http://www.vor.nu/langlab/bibliog.html}, 1992, agg. 1997.
%
\end{thebibliography}
% 
\end{document}


 
